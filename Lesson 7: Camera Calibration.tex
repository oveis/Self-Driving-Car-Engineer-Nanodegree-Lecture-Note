\documentclass[12pt]{article}
\usepackage[utf8]{inputenc}
\usepackage[legalpaper, margin=0.5in]{geometry}
\usepackage{listings}
\usepackage{xcolor}
\usepackage{graphicx}

\setlength{\parindent}{0em}
\setlength{\parskip}{1em}

%New colors defined below
\definecolor{codegreen}{rgb}{0,0.6,0}
\definecolor{codegray}{rgb}{0.5,0.5,0.5}
\definecolor{codepurple}{rgb}{0.58,0,0.82}
\definecolor{backcolour}{rgb}{0.95,0.95,0.92}

%Code listing style named "mystyle"
\lstdefinestyle{mystyle}{
  backgroundcolor=\color{backcolour},   
  commentstyle=\color{codegreen},
  keywordstyle=\color{magenta},
  numberstyle=\tiny\color{codegray},
  stringstyle=\color{codepurple},
  basicstyle=\ttfamily\normalsize,
  breakatwhitespace=false,         
  breaklines=true,                 
  captionpos=b,                    
  keepspaces=true,                 
  numbersep=5pt,                  
  showspaces=false,                
  showstringspaces=false,
  showtabs=false,                  
  tabsize=2
}

%"mystyle" code listing set
\lstset{style=mystyle}

\graphicspath{{resources/}}

\title{Lesson 7: Camera Calibration}
\author{Jinil Jang }
\date{October 2019}

\begin{document}

\maketitle

\section{Distortion}

\subsection{Radial distortion}

Real cameras use curved lenses to form an image, and light rays often bend a little too much or too little at the edges of these lenses. This creates an effect that distorts the edges of images, so that lines or objects appear more or less curved than they actually are.

\[x_{distorted} = x_{ideal}(1 + k_1r^2 + k_2r^4 + k_3r^6)\]
\[y_{distorted} = y_{ideal}(1 + k_1r^2 + k_2r^4 + k_3r^6)\]

, where \textbf{r} is the known distance between a point in an undistorted (corrected) image and the center of the image distortion, which is often the center of that image $(x_c, y_c)$.

\begin{figure}[htp]
    \centering
    \includegraphics[width=15cm]{distortion.png}
    \label{fig:distortion}
\end{figure}

\subsection{Tangential distortion}

This occurs when a camera’s lens is not aligned perfectly parallel to the imaging plane, where the camera film or sensor is. This makes an image look tilted so that some objects appear farther away or closer than they actually are.

\[x_{corrected} = x + [2p_1xy + p_2(r^2 + 2x^2)]\]
\[y_{corrected} = y + [p_1(r^2 + 2y^2) + 2p_2xy]\]

\section{Finding Corners}

\begin{lstlisting}[language=Python]
import numpy as np
import cv2
import matplotlib.pyplot as plt
import matplotlib.image as mpimg

# prepare object points
nx = 8 #Number of inside corners in x
ny = 6 #Number of inside corners in y

# Make a list of calibration images
fname = 'calibration_test.png'
img = cv2.imread(fname)

# Convert to grayscale
gray = cv2.cvtColor(img, cv2.COLOR_BGR2GRAY)

# Find the chessboard corners
ret, corners = cv2.findChessboardCorners(gray, (nx, ny), None)

# If found, draw corners
if ret == True:
    # Draw and display the corners
    cv2.drawChessboardCorners(img, (nx, ny), corners, ret)
    plt.imshow(img)
\end{lstlisting}

\section{Undistort}

Extract object points and image points for camera calibration.
\begin{lstlisting}[language=Python]
import numpy as np
import cv2
import glob
import matplotlib.pyplot as plt
%matplotlib qt

# prepare object points, like (0,0,0), (1,0,0), (2,0,0) ....,(6,5,0)
objp = np.zeros((6*8,3), np.float32)
objp[:,:2] = np.mgrid[0:8, 0:6].T.reshape(-1,2)

# Arrays to store object points and image points from all the images.
objpoints = [] # 3d points in real world space
imgpoints = [] # 2d points in image plane.

# Make a list of calibration images
images = glob.glob('calibration_wide/GO*.jpg')

# Step through the list and search for chessboard corners
for idx, fname in enumerate(images):
    img = cv2.imread(fname)
    gray = cv2.cvtColor(img, cv2.COLOR_BGR2GRAY)

    # Find the chessboard corners
    ret, corners = cv2.findChessboardCorners(gray, (8,6), None)

    # If found, add object points, image points
    if ret == True:
        objpoints.append(objp)
        imgpoints.append(corners)

        # Draw and display the corners
        cv2.drawChessboardCorners(img, (8,6), corners, ret)
        #write_name = 'corners_found'+str(idx)+'.jpg'
        #cv2.imwrite(write_name, img)
        cv2.imshow('img', img)
        cv2.waitKey(500)

cv2.destroyAllWindows()
\end{lstlisting}

Calculate distortion coefficients, and undistortion on an image.
\begin{lstlisting}[language=Python]
import pickle
%matplotlib inline

# Test undistortion on an image
img = cv2.imread('calibration_wide/test_image.jpg')
img_size = (img.shape[1], img.shape[0])

# Do camera calibration given object points and image points
ret, mtx, dist, rvecs, tvecs = cv2.calibrateCamera(objpoints, imgpoints, img_size,None,None)


dst = cv2.undistort(img, mtx, dist, None, mtx)
cv2.imwrite('calibration_wide/test_undist.jpg',dst)

# Save the camera calibration result for later use (we won't worry about rvecs / tvecs)
dist_pickle = {}
dist_pickle["mtx"] = mtx
dist_pickle["dist"] = dist
pickle.dump( dist_pickle, open( "calibration_wide/wide_dist_pickle.p", "wb" ) )
#dst = cv2.cvtColor(dst, cv2.COLOR_BGR2RGB)
# Visualize undistortion
f, (ax1, ax2) = plt.subplots(1, 2, figsize=(20,10))
ax1.imshow(img)
ax1.set_title('Original Image', fontsize=30)
ax2.imshow(dst)
ax2.set_title('Undistorted Image', fontsize=30)
\end{lstlisting}

\begin{figure}[htp]
    \centering
    \includegraphics[width=15cm]{camera_calibration.png}
    \label{fig:camera_calibration}
\end{figure}

\section{Undistort and Transform}

\begin{lstlisting}[language=Python]
import pickle
import cv2
import numpy as np
import matplotlib.pyplot as plt
import matplotlib.image as mpimg

# Read in the saved camera matrix and distortion coefficients
# These are the arrays you calculated using cv2.calibrateCamera()
dist_pickle = pickle.load( open( "wide_dist_pickle.p", "rb" ) )
mtx = dist_pickle["mtx"]
dist = dist_pickle["dist"]

# Read in an image
img = cv2.imread('test_image2.png')
nx = 8 # the number of inside corners in x
ny = 6 # the number of inside corners in y

# MODIFY THIS FUNCTION TO GENERATE OUTPUT 
# THAT LOOKS LIKE THE IMAGE ABOVE
def corners_unwarp(img, nx, ny, mtx, dist):
    # Pass in your image into this function
    # Write code to do the following steps
    # 1) Undistort using mtx and dist
    img = cv2.undistort(img, mtx, dist, None, mtx)
    
    # 2) Convert to grayscale
    gray = cv2.cvtColor(img, cv2.COLOR_BGR2GRAY)
    
    # 3) Find the chessboard corners
    ret, corners = cv2.findChessboardCorners(gray, (nx, ny), None)
    
    # 4) If corners found: 
    if ret == True:
        # a) draw corners
        cv2.drawChessboardCorners(img, (nx, ny), corners, ret)
        
        # b) define 4 source points src = np.float32([[,],[,],[,],[,]])
        src = np.float32(
            [[436, 114],
             [573, 137],
             [575, 268],
             [441, 256]])
        
        # c) define 4 destination points dst = np.float32([[,],[,],[,],[,]])
        dst = np.float32(
            [[100, 100],
             [242, 100],
             [242, 242],
             [100, 242]])
             
        # d) use cv2.getPerspectiveTransform() to get M, the transform matrix
        M = cv2.getPerspectiveTransform(src, dst)
        
        # e) use cv2.warpPerspective() to warp your image to a top-down view
        img_size = (img.shape[1], img.shape[0])
        warped = cv2.warpPerspective(img, M, img_size, flags=cv2.INTER_LINEAR)
        
    #delete the next two lines
    return warped, M

top_down, perspective_M = corners_unwarp(img, nx, ny, mtx, dist)
f, (ax1, ax2) = plt.subplots(1, 2, figsize=(24, 9))
f.tight_layout()
ax1.imshow(img)
ax1.set_title('Original Image', fontsize=50)
ax2.imshow(top_down)
ax2.set_title('Undistorted and Warped Image', fontsize=50)
plt.subplots_adjust(left=0., right=1, top=0.9, bottom=0.)
\end{lstlisting}

\begin{figure}[htp]
    \centering
    \includegraphics[width=15cm]{undistorted_warped_img.png}
    \label{fig:undistorted_warped_img}
\end{figure}

\end{document}