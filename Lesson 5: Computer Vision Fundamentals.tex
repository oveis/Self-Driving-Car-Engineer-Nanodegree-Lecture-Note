\documentclass[12pt]{article}
\usepackage[utf8]{inputenc}
\usepackage[legalpaper, margin=0.5in]{geometry}
\usepackage{listings}
\usepackage{xcolor}

\setlength{\parindent}{0em}
\setlength{\parskip}{1em}

%New colors defined below
\definecolor{codegreen}{rgb}{0,0.6,0}
\definecolor{codegray}{rgb}{0.5,0.5,0.5}
\definecolor{codepurple}{rgb}{0.58,0,0.82}
\definecolor{backcolour}{rgb}{0.95,0.95,0.92}

%Code listing style named "mystyle"
\lstdefinestyle{mystyle}{
  backgroundcolor=\color{backcolour},   commentstyle=\color{codegreen},
  keywordstyle=\color{magenta},
  numberstyle=\tiny\color{codegray},
  stringstyle=\color{codepurple},
  basicstyle=\ttfamily\normalsize,
  breakatwhitespace=false,         
  breaklines=true,                 
  captionpos=b,                    
  keepspaces=true,                 
  numbersep=5pt,                  
  showspaces=false,                
  showstringspaces=false,
  showtabs=false,                  
  tabsize=2
}

%"mystyle" code listing set
\lstset{style=mystyle}

\title{Lesson 5: Computer Vision Fundamentals}
\author{Jinil Jang }
\date{October 2019}

\begin{document}

\maketitle

\section{Canny Edge Detection}

\begin{lstlisting}[language=Python]
#doing all the relevant imports
import matplotlib.pyplot as plt
import matplotlib.image as mpimg
import numpy as np
import cv2

# Read in the image and convert to grayscale
image = mpimg.imread('exit-ramp.jpg')
gray = cv2.cvtColor(image, cv2.COLOR_RGB2GRAY)

# Define a kernel size for Gaussian smoothing / blurring
# Note: this step is optional as cv2.Canny() applies a 3x3 Gaussian internally
kernel_size = 3
blur_gray = cv2.GaussianBlur(gray,(kernel_size, kernel_size), 0)

# Define parameters for Canny and run it
low_threshold = 1
high_threshold = 10
edges = cv2.Canny(blur_gray, low_threshold, high_threshold)

# Display the image
plt.imshow(edges, cmap='Greys_r')
\end{lstlisting}

\section{Hough Transform}
Conversion from image space to Hough space.
Represent "x vs. y" line as a point in "m vs. b" where
\[y = mx + b\]

\begin{lstlisting}[language=Python]
# Define the Hough transform parameters
# Make a blank the same size as our image to draw on
rho = 1
theta = np.pi/180
threshold = 1
min_line_length = 10
max_line_gap = 1
line_image = np.copy(image)*0 #creating a blank to draw lines on

# Run Hough on edge detected image
lines = cv2.HoughLinesP(edges, rho, theta, threshold, np.array([]),
                            min_line_length, max_line_gap)

# Iterate over the output "lines" and draw lines on the blank
for line in lines:
    for x1,y1,x2,y2 in line:
        cv2.line(line_image,(x1,y1),(x2,y2),(255,0,0),10)

# Create a "color" binary image to combine with line image
color_edges = np.dstack((edges, edges, edges)) 

# Draw the lines on the edge image
combo = cv2.addWeighted(color_edges, 0.8, line_image, 1, 0) 
plt.imshow(combo)
\end{lstlisting}

\section{Finding Lines}
\begin{enumerate}
    \item Gray Image
    \item Gaussian Blur
    \item Canny
    \item Region of Interest
    \item Hough Lines
\end{enumerate}

\end{document}


